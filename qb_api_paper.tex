\documentclass{llncs}
\usepackage{graphicx}
\usepackage{listings}

\begin{document}

\title{Linked Open Statistical Data API: requirements and design criteria}

\author{xxx ddd \and yyy sss}

\maketitle

\begin{abstract}

\end{abstract}

\section{Introduction}\label{sec:intro}

Recently, many governments, organisations and companies are opening up their data for others to reuse through \textit{Open Data} portals  \cite{Kalampokis:2011:IJWET}. These data can be exploited to create added value services, which can increase transparency, contribute to economic growth and provide social value to citizens \cite{Janssen:2012}.

A major part of open data concerns statistics (e.g. economical and social indicators) \cite{Capadisli:2013}. These data are are often organised in a multidimensional way, where a measured fact is described based on a number of dimensions. In this case, statistical data are compared to data cubes. Thus, we onwards refer to statistical multidimensional data as \textit{data cubes} or just \textit{cubes}.

Linked data has been introduced as a promising paradigm for opening up data because it facilitates data integration on the Web \cite{Bizer:2009}. Concerning statistical data, the RDF data cube (QB) vocabulary enables modelling data cubes as linked data \cite{Cyganiak:2014:W3C}. In this way it facilitates their integration. Data provided using the QB vocabulary can be accessed using the existing machinery of Linked Data. However, skills and tooling for use of Linked Data (e.g. RDF, SPARQL) are less widespread than some other web technologies (e.g. REST, JSON). For example, there are many visualization libraries that consume data in JSON format (e.g D3.js, charts.js), while there are just a few that consume RDF. That's one of the reasons that there are not so many application for linked open statistical data [REF???].

Moreover, many portals that use the QB vocabulary often adopt different publishing practices \cite{KalampokisChallenges}, thus hampering their interoperability. As a result it is not easy to create generic software tools that operate across linked open statistical datasets. Usually, case specific software are created which assume that linked statistical data are published in a specific way. 

In this paper we describe the requirements and design criteria of an API that standardizes the interaction (i.e. input and output) with Linked Open Statistical Data in a way that facilitated the development of generic software. In this way we keep the advantages of linked data (e.g. data integration) but hide all the complexity by supporting developers to use linked statistical data stored in the form of an RDF Data Cube, while assuming minimal knowledge of Linked Data technologies. Moreover, the API offers a uniform way to access the underlying data hiding any data discrepancies, thus enabling the development of generic software tools that operate across datasets. 

\section{Methodology}\label{sec:methodology}

Related work:
\begin{itemize}
\item  OLAP APIs – interaction with multidimensional data (input): Oracle OLAP API [1], Olap4j [2], ++
\item Standardization of outcome: Json-stat, Json-ld, ++
\end{itemize}

Discussion with developers: Workshop, +++

\section{Solution overview}\label{sec:overview}

The JSON-QB API purpose is to help users retrieve information to support visualizations and other applications. It can be easily done as its implementation fills the gap of a generic software interaction with every type of LOSD. Moreover, it retains the benefits of Linked Data such as data representation and integration, but hides most of the complexity of that for the majority of users. Next, we present an overview of the solution that was designed.

The architecture of the API is relatively simple as it is developed on the RDF Data Cube vocabulary and SPARQL using statistical data from the data cube structure. The implementation of JSON-API abolishes the need to implement different data access layers for each tool is created. In the traditional architecture data access layers had to be coded separately leading to additional costs. The JSON-API can be installed on top of any RDF repository and offers basic and advanced operations on RDF Data cubes while other kinds of database could be used, enabling flexibility and innovation.

Linked data is a good approach for standards-based publication of statistics on the web, but RDF and SPARQL are unfamiliar to many users. There are obstacles to uptake of this technology because it is perceived to be complicated. The aim of the API is to support a style of interaction that is familiar to web developers - delivering data in JSON format and using familiar styles of API call. Through SPARQL queries, JSON-API has full access at data cubes returning the asked data in the re-used format of JSON. Users can use JSON as input to the API for -get requests and receive data as output still in JSON format.

There are already methods available for downloading entire data cubes but people often want just small parts.  Whole cubes are often too big to be well-suited to interactive applications, and if the data updates frequently, then it's important for people to be able to retrieve up-to-date extracts of the data, rather than keeping their own copies of full datasets up to date. JSON-QB API implementation allows developers through JSON to take exactly the data that want.
 
In addition, we want to standardise the API specification and try to get broad support for it, so that many data publishers can provide data in a compatible way, making tools interoperable. As well as making data easier to consume, using the API as the main method of delivering machine readable extracts of data would remove or greatly reduce the need for data publishers to provide public SPARQL endpoints. By this way cost can be reduced and reliability of open services could be improved.

\begin{figure}
  \includegraphics{images/overview.jpg}
\caption{Solution overview}
\label{fig:overview}
\end{figure}


\section{Requirements and design criteria}\label{sec:reqs}

\begin{itemize}
\item need to know what datasets are available
\item need to know about structure to subset the observations
\item in order not to return everything, need to subset
\item don't necessarily need a n-array/ tabular response - array of observations is sufficient. can always get back to the table
\item Filtering
\item Multilinguality
\item Ordering \& paging
\item merging, aggregations
\item json-ld representation is sufficient for query and response format
\item ++
\end{itemize}

API functionality:
\begin{itemize}
\item GET dataset-metadata
\item GET dimensions
\item GET attributes
\item GET measures
\item GET dimension-values
\item GET attribute-values
\item GET dimension-levels
\item GET slice
\item GET table
\item GET cubes
\item GET aggregationSetcubes
\item GET create-aggregations
\item GET cubeOfAggregationSet
\end{itemize}

\cite{Janssen:2012}

possible example for slice/ observation-selection query:
\begin{verbatim} 
{ 
  "jqql:dataset": "scot:home-care-clients",
  "jqql:filter": {
"dimension:gender": "gender:male",
   "dimension:age": { "jqql:greater-than": 50 }
  },
  "jqql:order": {
    "dimension:refPeriod": { "jqql:order-predicate": "ui:sortPriority", "jqql:direction": "jqql:asc"}
  },
  "jqql:page": {
    "jqql:limit": 10,
    "jqql:offset": 0
  }
}
\end{verbatim}

output:
\begin{verbatim} 
{ "observations": [ 
	{ "Average Cost": "1182", 
   	  "Date": "1-1-2013", 
	  "Day": "Tuesday", 
	  "Number of crashes": "5",
	  "Time": "No available time",
      "Total Cost": "5908", 
	  "@id": http://id.mkm.ee/observation/1" }, 
	{ "Average Cost": "400",
	  "Date": "1-1-2013",
	  "Day": "Tuesday",
	  "Number of crashes": "1",
	  "Time": "24:00",
 	  "Total Cost": "400",
	  "@id": "http://id.mkm.ee/observation/2" }
]}
\end{verbatim}

\section{Implementation}\label{sec:impl}

\section{Conclusion}\label{sec:conclusion}


%\begin{acknowledgements}
%If you'd like to thank anyone, place your comments here
%and remove the percent signs.
%\end{acknowledgements}

\bibliographystyle{splncs03}


\bibliography{qbbibfile}   % name your BibTeX data base


\end{document}


